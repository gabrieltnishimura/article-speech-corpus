\documentclass[conference]{IEEEtran}
\IEEEoverridecommandlockouts

\usepackage{multirow}
\usepackage[numbers]{natbib}
\usepackage[font=small]{caption}
\usepackage{graphicx}
\graphicspath{ {images/} }

\newcommand\cites[1]{\citeauthor{#1}'s\ (\citeyear{#1})}


\begin{document}
\title{What characterizes a speech corpus - a systematic literature review}
\author{\IEEEauthorblockN{Gabriel Takaoka Nishimura}
\IEEEauthorblockA{\textit{Polytechnic School of the University of S\~ao Paulo} \\
S\~ao Paulo, Brazil \\
gabriel.nishimura@usp.br}}
\maketitle

\begin{abstract}
Speech Corpora is an important base for research in the natural language processing field. In order to create new bases there is the need to identify what caracterizes a Speech Corpora. 
This document presents a Systematic Literature Review (SLR) for a better understanding of this gap. The SLR was conducted in the Web of Science Database and the papers found were analyzed and categorized. There were several findings in this SLR such as that most Corpora belong to the Read Speech Corpus class and that there is a significant number of dialect and multilingual corpora research. The findings in this paper can aid the future creation Speech Corporas in a structured and thought out way. 
\end{abstract}

\begin{IEEEkeywords}
speech corpus, systematic literature review, corpus creation process, natural language processing
\end{IEEEkeywords}

\section[Introduction]{Introduction}

Natural language processing is a vast field that explores how computers can understand and manipulate human language in text or speech format. Researches in this area includes (but are not limited to) sentiment analysis, sentence prediction, text translation, text to speech conversion and voice recognition. Voice recognition in particular, is a class of machine learning that can be stochastically modelled, using Hidden Markov Models \cite{gales2008application}, Neural Networks \cite{graves2013speech} or even non-stochastically \cite{burget2003nonrandomattr}.

However, these solutions degrades according to the data used in training stages. Factors such as noisy speech data, non-homogeneous recordings, different microphones within the same dataset and even speech disorders could limit proper analysis, affect accuracy and even change speech predictions. 

There are many robust datasets available in the literature, such as the TIMIT \cite{Lamel1992timmit}, DIRHA \cite{Ravanelli2016dirha} and the more recent \cite{chanchaochai2018globaltimit}. These datasets are called Speech Corpora and have a collection of audio recordings of spoken language. Some of them also have additional text files containing transcriptions of the words spoken. Nevertheless, most speech corpora are for the English language \cite{LeRouxVincent2014TRdatasets} and the literature on the Speech Corpora creation itself is sparse. 

The aim of this study is to identify the underlying characteristics of a speech corpus and the creation process itself. In order to better understand this gap, this work presents a systematic literature review. 

\section{Background}

\subsection{Automatic Speech Recognition}

Automatic Speech Recognition (ASR) is a independent, machine-based process of decoding and trasncribing oral speech \cite{asr2012levis}. A typical ASR system receives acoustic input from a speaker through a microphone, analyzes it using some pattern, model, or algorithm, and produces an output, usually in the  form of a text \cite{lai2008conversational}.

\subsection{Speech Corpus}

A Speech Corpus (plural form Speech Corpora) is a database of speech audio files and text transcriptions of these audio files. They generally fall into one of these two categories:

\begin{itemize}
    \item Read Speech
    
    This data is composed by scripted content, usually from books, broadcast news, list of words, sequence of numbers, etc. 

    \item Spontaneous Speech
    
    As for spontaneous speech corpora, gererated audio is not planned and may be recorded from dialogs, narratives \cite{ruhlemann2012introducing} and even map tasks (navigation instructions from one place to another) \cite{thompson1993hcrc}.
\end{itemize}

Many natural language processing areas can benefit from the creation of a speech corpora: speech synthesis, speech recognition, spoken language systems, speech recognition/verification \cite{gibbon1997handbook}.

\section[Related Work]{Related Work}

In this section, we discuss how the literature has treated speech corpora creation, as well as the various conditions and variables considered in the process.

Speech Corpus corpus creation itself is well stablished in the literature, by TIMIT \cite{Lamel1992timmit} and SWITCHBOARD \cite{godfrey1992switchboard}. TIMIT creates a dataset of 6300 utterances by 630 speakers from different regions of the United States. The sentences were crafted to fit in one of the three categories: 1) dialect "shibboleth", 2) phonemically-compact and 3) phonetically-diverse, but the selection itself was not well defined. Nevertheless, it is a very robust dataset with a time-aligned transcription and a usage guide to automatic speech recognition applications. 

The CHiME articles \cite{christensen2010chime} \cite{barker2013pascal}, \cite{barker2018fifth} (and more), are also source of structured speech corpora creation. Each of these articles challenge researchers to better recognize speech within a everyday listening environment using multiple distant microphones. Since the focus of these works lies on non-optimal recording conditions, detailed information on the noise background, noise level, recording style and speech material has been provided, as well as comprehensive post-processing work.

A more recent work by \cite{chanchaochai2018globaltimit} attempts to extend the TIMIT functionality to other languages, by providing a method to create "TIMIT-like" datasets. These datasets are caracterized by having 1) Multiple (anonymously) identified speakers, 2) Wide range of phonetically-representative inputs, 3) Wideband recordings with good acoustic quality, 4) Time-aligned lexical and phonemic transcripts and 5) Easily availability to anyone. The authors detail the speakers and sessions, the text corpus selection process, the recording procedures, as well as the transcriptions and alignment methods. At the moment, there have been five datasets created, with more planned or in progress.

As for works reviewing the literature, \cite{furui2005recent} focuses on the technical dificulties of creating a spontaneous speech corpus. Techniques such as sentence boundary detection, pronunciation modeling, acoustic as well as language model adaptation, and automatic summarization are some of the analysed in the time the article was released.

The work from \cite{LeRouxVincent2014TRdatasets}, reference for \cite{8423508}, \cite{Gannot:2017:CPM:3105654.3105658}, \cite{RODOMAGOULAKIS2017419}, compares speech corpora within realistic environments, defining a set of characteristics for each analyzed database: general attributes (scenario, total duration, sampling rate etc), speech (language, speaking style etc), channel (type, speaker etc), noise (type) and groud truth (positional reference, paralinguistic attributes etc).

\section[Research Method]{Research Method}

To identify the characteristics of a speech corpus, a systematic literature was performed \cite{kitchenham2009systematic}.

The search was executed on August 1st, 2019 and the electronic database used was Web of Science. Since corpora are also presented at conferences, all article types were considered in the search, not limited by publication date. The whole process to choose relevant papers is defined below:

\subsection{Search in papers database}

The first step in the process is searching for papers in the specified database. To allow reproducibility, the search query is presented in the table \ref{tab:search-terms}. No criteria were applied in the search, therefore considering the title, abstract, keywords and the entire paper for indexing. It is also noted that conference proceedings are also included in this search. In this step, 157 results were yieded. This result was further filtered to 23, which contained the terms in the title.

\begin{table}[h]
    \centering
    \begin{tabular}{|c|c|}
        \hline Digital Library & Search terms \\ \hline
        Web of Science & TOPIC: ("speech corpus" OR "speech corpora")  \\ \hline
    \end{tabular}
    \caption{Search terms used in the SLR}
    \label{tab:search-terms}
\end{table}

\subsection{Inclusions and exclusions}

In order to filter relevant results, we define the criteria to include or exclude the articles based on each abstract read:

\subsubsection{Inclusion Criteria}

\begin{itemize}
    \item Creates a speech corpus
    \item Define or discuss speech corpus creation
\end{itemize}

\subsubsection{Exclusion Criteria}

\begin{itemize}
    \item Full text not available in electronic document
    \item Does not contain "speech corpus" or "speech corpora" string in title, abstract or keywords.
\end{itemize}

This step reduced the number of articles to 14.

\subsection{Article Analysis}

After applying the proper filtering as defined by the previous section, the remaining articles can be fully read to find the results. The whole filtering process is summaryzed in \ref{tab:filtering}.

\begin{table}[h]
    \centering
    \begin{tabular}{|c|c|}
        \hline Step & Results \\ \hline
        Initial search & 157 \\ \hline
        Title filtering & 23 \\ \hline
        Abstract reading & 14 \\ \hline
    \end{tabular}
    \caption{SLR - Filtering of results}
    \label{tab:filtering}
\end{table}

\section{Findings}

In order to analytically organize our findings, a content-analysis approach was used. To choose the categories in which the content-analysis is applied, we adapted the work from \cite{queiroz2019blockchain}, resulting in \ref{tab:content-analysis}. As for the corpora categories, we sought a comparison work from  \cite{LeRouxVincent2014TRdatasets}, in which a comprehensive table was provided and adapted to fit our categorization needs.

\begin{table}[h]
    \centering
    \begin{tabular}{|p{4cm}|p{4cm}|}
        \hline Category & Explanation/Example \\ \hline
        Total number of papers published & Number of publications by year \\ \hline
        Application area & Speech Corpus, Learning English Pronunciation, Automatic Speech Recognition, Crowdsourcing \\ \hline
        Application context & Speech Corpus Creation, Speech Corpus Automated Analysis, Dialect Analysis \\ \hline
        Corpus Language & English (Dialect), Arabic, Multilingual, etc \\ \hline
    \end{tabular}
    \caption{Categories for content analysis}
    \label{tab:content-analysis}
\end{table}

\begin{table}[h]
    \centering
    \begin{tabular}{|p{2cm}|p{6cm}|}
        \hline Category & Explanation/Example \\ \hline
        General attributes & scenario, total duration, sampling rate, number of distant or noisy microphones, demographics, transcription and time alignment and open access \\ \hline
        Speech attributes & duration of speech, speeaking style, speakers in the room, speaker overlap\\ \hline
        Channel attributes & channel type, speaker location, speaker movments \\ \hline
        Noise attributs & stationary background noise, car noise, meeting noisees, domestic noises, outdoor noises \\ \hline
        Available ground truth & reference speech signal, speaker location and orientation, paralinguistic attributes, noise events \\ \hline
    \end{tabular}
    \caption{Categories for speech corpus construction}
    \label{tab:speech-analysis}
\end{table}

\subsection{Publication by country and year}

Table \ref{tab:country-analysis} reports the number of papers, published by country and year, resulting from the research-protocol application.

\begin{table}[h]
    \centering
    \begin{tabular}{|c|c|c|c|c|c|}
        \hline Country & 2005-2015 & 2016 & 2017 & 2018 & (\%) \\ \hline
        United States & - & 3 & 2 & - & 33\% \\ \hline 
        Japan & 3 & - & - & - & 20\% \\ \hline
        Emirates Arabs & - & - & 1 & 1 & 13\% \\ \hline
        India & - & - & 1 & - & 6.7\% \\ \hline
        Malasya & - & - & 1 & - & 6.7\% \\ \hline
        Sweden & - & - & 1 & - & 6.7\% \\ \hline
        China & - & 1 & - & - & 6.7\% \\ \hline
        Poland & 1 & - & - & - & 6.7\% \\ \hline
    \end{tabular}
    \caption{Countries for content analysis}
    \label{tab:country-analysis}
\end{table}

\subsection{Publication by type}

As for publication types, they have almost been equaly divided: 46.7\% for articles and 53.3\% for proceedings papers.

\subsection{Speech Corpus studies' categorization}

In table \ref{tab:results-categorization}, each article is categorized according to table \ref{tab:content-analysis} definitions. The analysis is done afterwards in section \ref{sec:discussion}.

\begin{table*}[h]
    \centering
    \begin{tabular}{|l|p{3cm}|p{6.5cm}|p{3cm}|}
        \hline Authors & Application area & Context & Language \\ \hline
        \cites{almeman2018building} & Read Corpus & Multi-dialect Arabic corpora is scarse and not generated on mobile & Arabic \\ \hline
        \cites{dwivedi2017documenting} & Language Documentation & Revitalization of endangered languages through corpus creation & Kanauji of Kanpur \\ \hline
        \cites{bougrine2017altruistic} & Crowdsourcing & Crowdsourcing is a emerging and collaborative approach and can be effectively used to annotate linguistic resources & Arabic Algerian Dialects \\ \hline
        \cites{ng2017shefce} & Automatic sillable and phoneme detection & Pronunciation assessment studies in a bilingual context & Billingual (Cantonese, English) \\ \hline
        \cites{moore2017sheffield} & Read Speech & Goal oriented conversation & British English with a southern accent \\ \hline
        \cites{ramli2017first} & Storytelling Speech & Under-resourced language corpus creation for humanoid robot storyteller & Malay \\ \hline
        \cites{mansikkaniemi2017automatic} & Automatic speech alignment & Transcribed speech is a scarce and expensive resource & Finish \\ \hline
        \cites{goldman2016siwis} & Cross language speaker adaptation & Cross-lingual studies have no speech corpus from the same speakers & Bilingual and Trilingual from (English, French, German and Italian) \\ \hline
        \cites{liu2016sheffield} & Spontaneous Speech & Distant speech recognition with multi-channel speech corpus & English \\ \hline
        \cites{ruilan2016improving} & Read Corpus & Pronunciation characteristics in non-native english speakers & Dialectal English \\ \hline
        \cites{klessa2013paralingua} & Read Corpus & Paralinguistic features detection for forensics & Paralinguistic \\ \hline
        \cites{nagino2008building} & Corpus Enhancement & Low cost speech corpus creation with statistical multidimensional scaling method & Japanese \\ \hline
        \cites{zhang2008improved} & Phonetic words selection & Phonetically rich word selection from larger corpus & Chinese \\ \hline
        \cites{clopper2006nationwide} & Read Corpus & Large amount of speech by male and female from six dialeect regions & Multiple-dialect English \\ \hline
    \end{tabular}
    \caption{Findings organized by speech corpus categorization as per table \ref{tab:content-analysis}}
    \label{tab:results-categorization}
\end{table*}

\begin{table*}[h]
\centering
\begin{tabular}{|l|l|l|l|}
\hline
Category & Characteristic & Work(s) & Quantity (out of 14)  \\ \hline
\multirow{7}{*}{General attributes} 
    & Scenario & 
    \cite{almeman2018building}, \cite{dwivedi2017documenting}, \cite{bougrine2017altruistic}, \cite{bougrine2017altruistic}, \cite{moore2017sheffield}, \cite{ramli2017first}, \cite{goldman2016siwis}, \cite{liu2016sheffield}, \cite{ruilan2016improving}, \cite{klessa2013paralingua}, \cite{nagino2008building},  \cite{clopper2006nationwide}, \cite{zhang2008improved} & 13
    \\ \cline{2-4} & Total duration &
    \cite{dwivedi2017documenting}, \cite{bougrine2017altruistic}, \cite{bougrine2017altruistic}, \cite{moore2017sheffield}, \cite{ramli2017first}, \cite{goldman2016siwis}, \cite{liu2016sheffield}, \cite{ruilan2016improving}, \cite{klessa2013paralingua}, \cite{nagino2008building}, \cite{clopper2006nationwide}, \cite{zhang2008improved} & 12
    \\ \cline{2-4} & Sampling Rate &
    \cite{almeman2018building}, \cite{dwivedi2017documenting}, \cite{bougrine2017altruistic}, \cite{ng2017shefce}, \cite{moore2017sheffield}, \cite{ramli2017first}, \cite{goldman2016siwis}, \cite{liu2016sheffield}, \cite{ruilan2016improving}, \cite{klessa2013paralingua}, \cite{clopper2006nationwide} & 11
    \\ \cline{2-4}& Distant or noisy microphones & 
    \cite{moore2017sheffield}, \cite{ramli2017first}, \cite{liu2016sheffield} & 3
    \\ \cline{2-4} & Demographics &
    \cite{dwivedi2017documenting}, \cite{bougrine2017altruistic}, \cite{ng2017shefce}, \cite{goldman2016siwis}, \cite{ruilan2016improving}, \cite{klessa2013paralingua}, \cite{clopper2006nationwide} & 7
    \\ \cline{2-4} & Transcription and time alignment & % manual?
    \cite{dwivedi2017documenting}, \cite{bougrine2017altruistic}, \cite{bougrine2017altruistic}, \cite{ng2017shefce}, \cite{moore2017sheffield}, \cite{ramli2017first}, \cite{goldman2016siwis}, \cite{liu2016sheffield}, \cite{ruilan2016improving}, \cite{klessa2013paralingua}, \cite{nagino2008building}, \cite{clopper2006nationwide} & 12
    \\ \cline{2-4} & Open access &
    \cite{almeman2018building}, \cite{moore2017sheffield}, \cite{liu2016sheffield} & 3
\\ \hline \multirow{3}{*}{Speech attributes}
    & Unique words & 
    \cite{bougrine2017altruistic}, \cite{bougrine2017altruistic}, \cite{ng2017shefce}, \cite{moore2017sheffield}, \cite{ramli2017first}, \cite{goldman2016siwis}, \cite{liu2016sheffield}, \cite{ruilan2016improving}, \cite{klessa2013paralingua}, \cite{nagino2008building}, \cite{clopper2006nationwide} & 11
    \\ \cline{2-4} & Speaking style &
    \cite{dwivedi2017documenting}, \cite{bougrine2017altruistic}, \cite{ng2017shefce}, \cite{moore2017sheffield}, \cite{ramli2017first}, \cite{goldman2016siwis}, \cite{nagino2008building}, \cite{clopper2006nationwide} & 8
    \\ \cline{2-4} & Number of speakers & 
    \cite{dwivedi2017documenting}, \cite{bougrine2017altruistic}, \cite{bougrine2017altruistic}, \cite{ng2017shefce}, \cite{moore2017sheffield}, \cite{ramli2017first}, \cite{goldman2016siwis},, \cite{ruilan2016improving} \cite{liu2016sheffield}, \cite{klessa2013paralingua}, \cite{nagino2008building}, \cite{clopper2006nationwide} & 14
\\ \hline \multirow{3}{*}{Channel attributes}
    & Channel type & 
    \cite{dwivedi2017documenting}, \cite{bougrine2017altruistic}, \cite{ng2017shefce}, \cite{moore2017sheffield}, \cite{ramli2017first}, \cite{goldman2016siwis}, \cite{liu2016sheffield}, \cite{ruilan2016improving}, \cite{nagino2008building}, \cite{clopper2006nationwide} & 10
    \\ \cline{2-4} & Speaker location & 
    \cite{dwivedi2017documenting}, \cite{ng2017shefce}, \cite{liu2016sheffield} & 3
    \\ \cline{2-4} & Speaker movements & 
    \cite{dwivedi2017documenting}, \cite{moore2017sheffield}, \cite{liu2016sheffield} & 3
\\ \hline \multirow{1}{*}{Noise attributes}
    & Noise type & N/A &
    0
\\ \hline \multirow{4}{*}{Available ground truth} 
    & Reference speech signal & 
    \cite{ng2017shefce}, \cite{moore2017sheffield}, \cite{ramli2017first}, \cite{liu2016sheffield} & 4
    \\ \cline{2-4} & Speaker location and orientation &
    \cite{bougrine2017altruistic}, \cite{moore2017sheffield}, \cite{ramli2017first}, \cite{liu2016sheffield} & 4
    \\ \cline{2-4} & Paralinguistic attributes &
    \cite{klessa2013paralingua} & 1
    \\ \cline{2-4} & Noise events & 
    \cite{ramli2017first} & 1
    \\ \hline
\end{tabular}
\caption{Speech corpora characteristics defined in table \ref{tab:speech-analysis}}
\label{tab:results-attributes}
\end{table*}

\section{Discussion}
\label{sec:discussion}

Each of our findings is discussed on the following subsections.

\subsection{Publications by country and year}

A simple analysis on table \ref{tab:country-analysis} infer that the research on speech corpus creation was stable until 2018, which had its last article published. This can be explained by the limited number of "Open Access" articles found and the small number of databases searched (Web Of Science only). A more permissive literature review should be able to better understand the conditions on which non-open articles are being released, as well as enhance coverage of the state-of-the-art in this field.

The table also categorizes the countries research quota. More developed countries such as United States and Japan have created more corpora over the years.

\subsection{Publications by type}

In the search executed, the remaining articles were almost equally separeated by type. A great number of conference proceedings may indicate that the corpus creation process is not properly defined, as works vary in content and structure. Again, the main limitation of this work is the only database searched, therefore skewing these analysis.

\subsection{Speech corpus categorization}

Table \ref{tab:content-analysis} contains the content-analysis for the works in the systematic literature review. Two of the colums - Application Area (1), and Language (2) - are be discussed below:

\subsubsection{Application Area}

Most of the corpora found in the articles were a Read Speech Corpus (\cite{almeman2018building}, \cite{ruilan2016improving}, \cite{klessa2013paralingua}, \cite{clopper2006nationwide}), revealing the lack of a structured language documentation in current research. This can be explained by the type of it's content: a scripted content, with less bias from the speakers when compared to spontaneous corpora. Three of these works (\cite{ng2017shefce}, \cite{mansikkaniemi2017automatic} and \cite{nagino2008building}) worked on the automating corpus creation by automatically aligning words, detecting sillables and phonemes and even enhancing corpora, as currently most recording and alignment work is done manually.

\subsubsection{Languages}

There is a significant number of dialect and multilingual corpora research in the analysed works. These variations of the same language - by different region, or by different pronunciation by non-native speakers -, suggest that speech technology could become more personalized as computers understand less traditional utterances of the same language.

\subsection{Works by speech corpus characteristics}

Our findings condensed at table \ref{tab:results-categorization} illustrate the
current speech corpus creation characteristics used in the literature, each discussed below:

\subsubsection{Scenario}

Almost all corpora specified the scenario at which the recordings were executed. A meeting room (\cite{liu2016sheffield}, \cite{moore2017sheffield}); a anechoic booth (\cite{goldman2016siwis}); a quiet room inside a laboratory (\cite{ramli2017first}), etc. Though, each corpus is not limited to one scenario. For instance, \cite{almeman2018building} recorded in four different scenarios: inside the home, in a moving car, in a public place and in a quied place. Such characteristic is critical to a corpus creation.

\subsubsection{Total duration}

12 out of the 14 studies analysed described the total duration of the recordings. This illustrates the importance of being descriptive in the corpus.

\subsubsection{Sampling Rate}

The sampling rate is major to the corpus creation. It is cited in 11 out of the 14 works read. The reason it has so much importance lies on the corpus main usage: automatic speech recognition software. Knowing the frequency at which the recordings are done, the computer is able to better recognize subtleties in a more sampled audio, as well as simplify processing in a lower frequency sample rate.

\subsubsection{Noisy microphones}

Non-optimal recording conditions are normal. For instance, speaking to the phone in a public space generates very high noisy background \cite{moore2017sheffield}). Also, speaking far from the microphone is also a common occurence. The presence of works in this category emphasize the need of noisy corpora to ensure automatic speech recognition software takes these factors into account.

\subsubsection{Demographics}

When recording accents (\cite{moore2017sheffield}), dialects (\cite{almeman2018building}) and pronunciations (\cite{ng2017shefce}),  definitions of the speakers demographics supports understanding of variations the same language through research.

\subsubsection{Transcription and time alignment}

In the context of automatic speech recognition, a speech corpus containing  transcription and time alignment (additional to the recordings) enable computers and speech recognition specialists a more faithful analysis. It's importance can be verified by 85\% of the works evaluated.

%\subsubsection{Open access}

%Most corpora are closed to the public viewer. 

%\subsubsection{Unique words}
%\subsubsection{Speaking style}
\subsubsection{Number of speakers}

As well as the total duration, the number of speakers is a major characteristic to the corpus. More speakers lead to more representativeness, but also affect the corpus recording time. It is a important parameter and every paper analysed describes it.

%\subsubsection{Channel type}
%\subsubsection{Speaker location}
%\subsubsection{Speaker movements}
%\subsubsection{Noise type}
%\subsubsection{Reference speech signal}
%\subsubsection{Paralinguistic attributes}
%\subsubsection{Noise events}

\section{Conclusion}

This study investigated what compromises a speech corpus and the characteristics it should contain. A systematic literature review was used to identify the research on the theme, as well as analyse each category. Our findings revealed a diverse number of application areas, different languages and their variations, and a list of characteristic - which each work got separated to -. A more comprehensive analysis allowed identification of common characteristics to a Corpus, and a better interpretation of when to apply more specific characteristics, such as distant or noisy microphones. As for the limitations of this research, it should be noted that the literature review was only applied using one database. Future work should be able to extend this review and search more databases, and also propose a method on how to create a well defined Speech Corpus.

\bibliographystyle{plainnat}
\bibliography{referencias}
\end{document}